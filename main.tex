\documentclass[11pt, letterpaper, oneside, titlepage]{article}
\usepackage{fontspec}
  \setmainfont{Hack}
\usepackage{geometry}
  \geometry{left=2cm, top=2.5cm, right=2cm, bottom=2cm} % Configure page margins with geometry
\usepackage{listings}
\usepackage{color}

%%%%%%%%%%%%%%%%%%%%%%%% Useful Macros %%%%%%%%%%%%%%%%%%%%%%%%%%%%%%%%%%%%%%%%%
%\newcommand{\commandLine}[1]{
%  \lstset{language=bash}
%  \begin{lstlisting}[frame=single]^^n
%    #1^^n
%  \end{lstlisting}
%}

\title{*nix Documentation \\\large Linux Hardeneing 101}
\author{ Xavier Loose \\ xav.loose@gmail.com \\ github.com/xavloose \and Zhixiu Lu \\ Zhixiu.Lu@coyotes.usd.edu }
\date{\today}

\begin{document}

  \maketitle
  %\commandLine{wget pussy}
  %-----------------------------------------------------------------------------
  % This is where text is imported. linuxSec/IntroTemplate.tex can be used as
  %   guide for developing a section
  %
  % Adding a section should follow the following format
  % \part{Name of Section}
  %   \input{relative/file/location/filename.tex}
  %   \input{relative/file/location/filename2.tex}
  %   ...
  %   \newpage
  %-----------------------------------------------------------------------------
  \part{Introduction}
    %%%%%%%%%%%%%%%%%%%%%%%%%%%%%%%%%%%%%%%%%%%%%%%%%%%%%%%%%%%%%%%%%%%%%%%%%%%%%%%%
% Author: Xavier Loose
% 2017
%
% This doubles as a template for creating a section for USD's CCDC Competetion
%
%%%%%%%%%%%%%%%%%%%%%%%%%%%%%%%%%%%%%%%%%%%%%%%%%%%%%%%%%%%%%%%%%%%%%%%%%%%%%%%%

% The section is defined in main.tex so we only need to create subsections
% There is no need to format becuase we have already done that.

% These represent the heiarchy for sectioning a larger section
% If you keep proper indentation it will make it easier to follow for easier for
% future members
% \subsection{name}
% \subsubsection{name}
% \paragraph{name}
% \subparagraph{name}

\section{Preface}
  This section is primarily inteded for Linux since that is what we will most likely be dealing with at CCDC, but it should be noted that given another Unix system should not cause panic as I'll describe. Linux was a fork of the Unix kernel and since then has grown in popularity and has become more common than it's parent. Since our primary goal is to stick to the basics and a lot of the software is compatible with all *nix systems then. Linux specialites will be prefaced with "Linux" and "Unix" for unix systems.

\section{How to Use This}
  This documentation is intended to keep the information short and sweet for the purposes of competetion. The effort has been applied so that it can be quickly analyzed in a highly stressful environment.

  \subsection{Identifying Commands}
  

    \input{linuxSec/ProtectionPlan.tex}
    \newpage
  \part{Phase 1}
    \input{linuxSec/Stage1.tex}
    \newpage
  \part{Phase 2}
    \input{linuxSec/Stage2.tex}
    \newpage
  \part{Phase 3}
    \input{linuxSec/Stage3.tex}
    \newpage
  \part{Linux Kernel}
    \input{linuxSec/KernelSecurity.tex}
    \input{linuxSec/InstallingAKernel.tex}
    \input{linuxSec/BuildingAKernel.tex}
    \newpage
  \part{Linux GRSecurity}
    %\newpage
  \part{Linux SELinux}
    %\newpage
  \part{ModSecurity}
    \input{linuxSec/ModSecurity.tex}
    \newpage
\end{document}
